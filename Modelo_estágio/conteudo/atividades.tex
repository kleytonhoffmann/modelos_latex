\chapter{ATIVIDADES DESENVOLVIDAS} %Capítulo sempre em LETRAS MAIÚSCULAS
\label{chap:atividades}

Mostrar as atividades desenvolvidas no período de estágio. Esta etapa tem que ser coerente ao que foi preenchido no formulário de acompanhamento das atividades desenvolvidas pelo estagiário, o qual deve ser, preferencialmente, preenchido semanalmente.


\section{Exemplo de Seção}
\label{sec:apresentacao}


\begin{equation}
\label{eq:2}
x = \frac{{ - b \pm \sqrt {{b^2} - 4\,ac} }}{{2a}}
\end{equation}

\begin{equation}
\label{eq:1}
\left[ {\begin{array}{*{20}{c}}
	1&0&0\\
	0&1&0\\
	0&0&1
	\end{array}} \right] = I
\end{equation}



A parcela $ \frac{-b}{2a}$ é ...

\subsection{Exemplo subSeção}

\subsubsection{Exemplo subsubsection}

Uma grande dica aqui é a utilização do site \textbf{www.tablesgenerator.com} para geração de tabelas.

\begin{table}[!htpb]
	\centering
	\caption{Título da tabela}
	\label{my-label}
	\begin{tabular}{lcc}
		& \multicolumn{1}{l}{Coluna 2} & \multicolumn{1}{l}{Coluna 2} \\ \hline
		Linha 1 & a                            & b                            \\ \hline
		Linha2  & c                            & d                           
	\end{tabular}
	
	
	\fonte{\citeonline{livro-unoesc}.}
	%  \nota{Exemplo de nota}
	% \nota[Anotações]{Exemplo nota personalizada}
\end{table}

\subsection{Imagens}


\begin{figure}[!htpb]
	\centering
	\caption{Motor de indução trifásico}
	\label{fig:Motores}
	\borda{\includegraphics[scale=0.8]{figuras/motores}}
	\fonte{\cite{NR10}.}
\end{figure}

\section{CITAÇÃO}

Segundo  \citeonline[p. 94]{livro-unoesc},
\begin{citacao}
	``Compreende trecho transcrito que apresenta mais de três linhas; mantém-se o discurso do texto original; destaca-se em blocos, espaço simples,com recuo de 4 cm a partir da margem esquerda, com letra menor que a do texto original;sugere-se usar tamanho 10''.
\end{citacao}

Utilizando outros comando de citação \cite{fitzgerald2014}. Teste de citeonline, segundo \citeonline{LUCKMANN2008}............
